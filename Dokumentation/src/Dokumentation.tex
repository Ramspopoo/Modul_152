%! Author = hilfiker
%! Date = 21.07.2022

% Preamble
\documentclass[11pt]{article}

% Packages
\usepackage{amsmath} % Formulas in document
\usepackage{pdflscape} % Landscaped pages in Pdf
\usepackage[a4paper, margin=0.8in]{geometry} % Set the margin and size of a page
\usepackage[hidelinks]{hyperref} % Remove Boxes around Hyperlinks
\usepackage{lastpage} % Custom page numbering
\usepackage{pgfgantt} % Gantt-Tables
\usepackage{fancyhdr}
\usepackage{graphicx}
\usepackage{xspace}
\usepackage{mwe}
\usepackage{caption}
\usepackage{tcolorbox} % Custom page numbering

\usepackage{titling}
\usepackage{textcomp}
\usepackage{wasysym}
\usepackage[utf8]{inputenc}
\usepackage[T1]{fontenc}
\usepackage{enumerate}
\renewcommand\maketitlehooka{\null\mbox{}\vfill}
\renewcommand\maketitlehookd{\vfill\null}

\renewcommand*\contentsname{Inhaltsverzeichnis}
\renewcommand{\figurename}{Abb.}

% Configure page numbering & footer
\pagestyle{fancy}
\fancyhf{}
\lfoot{Tobias Hilfiker}
\cfoot{Seite \thepage \hspace{1pt} von \pageref{LastPage}}
\rfoot{\today}

% Methadata
\title{
    \includegraphics[width=\textwidth]{media/curling_logo}
    \begin{center}
        Modul 152 \\
        Multimediainhalte in einen Webauftritt integrieren\\
        E-Portfolio
    \end{center}}
\author{Tobias Hilfiker}
\date{\today}

% Document
\begin{document}

    %Kapitel Frontpage
    \begin{titlingpage}
        \maketitle
    \end{titlingpage}
    \pagebreak

    %Kapitel Table of contents
    \tableofcontents
    \pagebreak

    %Kapitel 1 - Einleitung
    \section{Einleitung}
    Im vorgehenden Storyboard habe ich die Umsetzung der Webseite beschrieben, welche ich im Rahmen dieses Projektes umsetzen darf.

    %Kapitel 2 - Webseite
    \section{Webseite}

    %Kapitel 2.1 - Eingesetzte Technologien
    \subsection{Eingesetzte Technologien}

    %Kapitel 2.2 - Reflexion Webseite
    \subsection{Reflexion Webseite}

    %Kapitel 3 - Video
    \section{Video}

    %Kapitel 3.1 - Erstellung Video
    \subsection{Filmen Video}

    %Kapitel 3.2 - Schnitt des Videos
    \subsection{Schnitt des Videos}

    %Kapitel 3.3 - Reflexion Video
    \subsection{Reflexion Video}

    %Kapitel 4 - Bildmanipulationen
    \section{Bildmanipulationen}

    %Kapitel 4.1 - Bildmanipulation 1
    \subsection{Bildmanipulation 1}

    %Kapitel 4.2 - Bildmanipulation 2
    \subsection{Bildmanipulation 2}

    %Kapitel 4.3 - Bildmanipulation 3
    \subsection{Bildmanipulation 3}

    %Kapitel 4.4 - Reflexion Bildmanipulationen
    \subsection{Reflexion Bildmanipulationen}

    %Kapitel 5 - Fazit
    \section{Fazit}

\end{document}