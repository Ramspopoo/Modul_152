%! Author = hilfiker
%! Date = 21.07.2022

% Preamble
\documentclass[11pt]{article}

% Packages
\usepackage{amsmath} % Formulas in document
\usepackage{pdflscape} % Landscaped pages in Pdf
\usepackage[a4paper, margin=0.8in]{geometry} % Set the margin and size of a page
\usepackage[hidelinks]{hyperref} % Remove Boxes around Hyperlinks
\usepackage{lastpage} % Custom page numbering
\usepackage{fancyhdr}
\usepackage{graphicx}
\usepackage{tcolorbox} % Custom page numbering

% Configure page numbering & footer
\pagestyle{fancy}
\fancyhf{}
\lfoot{Tobias Hilfiker}
\cfoot{Seite \thepage \hspace{1pt} von \pageref{LastPage}}
\rfoot{\today}

% Methadata
\title{
    \includegraphics{media/frontpage.jpg}
    Modul 152 \\
    Multimediainhalte in einen Webauftritt integrieren}
\author{Tobias Hilfiker}
\date{\today}

% Document
\begin{document}

    %Kapitel Frontpage
    \maketitle
    \pagebreak

    %Kapitel Table of contents
    \tableofcontents
    \pagebreak

    %Kapitel 1 - Einleitung
    \section{Einleitung}
    Im Rahmen des Modul 152 darf ein E-Portfolio inklusive vorgehendem Storyboard erstellt werden.
    Dieses Storyboard soll eine Vorbereitung für das E-Portfolio sein, welches in einer separaten
    Dokumentation beschrieben wird.

    \\
    \textbf{Spezielles}\\
    Um das Storyboard und die DOkumentation zum E-Portfolio zu schreiben, verwende ich \latex. Das ist
    eine Skriptbasierte Dokumentationssprache, ähnlich wie Markdown. Primär gibt es zwei Unterschiede zu
    Markdown:

    %Kapitel 2 - Thema
    \section{Thema}

    todo kleiner Text

    %Kapitel 2.1 - Wahl des Themas
    \subsection{Wahl des Themas}

    Ich habe das Thema Curling ausgewählt, da es bereits seit vielen Jahren mein Hobby ist. Jeden Winter
    von ca. September bis April spiele ich in einem Juniorenteam der Liga C in St.Gallen.

    %Kapitel 3 - Visualisierung E-Portfolio
    \section{Visualisierung E-Portfolio}

    %Kapitel 3.1 - Menüstruktur
    \subsection{Menüstruktur}
    Als Menü-Navigation habe ich mich für eine Navigation am oberen Teil der Webseite

    %Kapitel 3.2 - Farbwahl
    \subsection{Farbwahl}

    %Kapitel 3.3 - Mockups
    \subsection{Mockups}
    - Mockups mit Adobe xd -> Wieso
    - Bilder der Mockups einfügen

    %Kapitel 4 - Bildbearbeitung
    \section{Bildbearbeitung}

    %Kapitel 4.1 - Bildbearbeitungsmethode
    \subsection{Bildbearbeitungsmethode}

    %Kapitel 5 - Logo
    \section{Logo}

    %Kapitel 5.1 - Idee
    \subsection{Idee}

    %Kapitel 6 - Technisches Konzept E-Portfolio
    \section{Technisches Konzept E-Portfolio}
    -Welche Technologien werden verwendet
    -

    %Kapitel 7 - Reflexion
    \section{Reflexion}

    %Kapitel 7.1 - Bildbearbeitung
    \subsection{Bildbearbeitung}

    %Kapitel 7.2 - Logodesign
    \section{Logodesign}


\end{document}