%! Author = hilfiker
%! Date = 21.07.2022

% Preamble
\documentclass[11pt]{article}

% Packages
\usepackage{amsmath} % Formulas in document
\usepackage{pdflscape} % Landscaped pages in Pdf
\usepackage[a4paper, margin=0.8in]{geometry} % Set the margin and size of a page
\usepackage[hidelinks]{hyperref} % Remove Boxes around Hyperlinks
\usepackage{lastpage} % Custom page numbering
\usepackage{fancyhdr}
\usepackage{graphicx}
\usepackage{tcolorbox} % Custom page numbering

\usepackage{titling}
\usepackage{textcomp}
\usepackage{wasysym}
\usepackage[utf8]{inputenc}
\usepackage[T1]{fontenc}
\renewcommand\maketitlehooka{\null\mbox{}\vfill}
\renewcommand\maketitlehookd{\vfill\null}

% Configure page numbering & footer
\pagestyle{fancy}
\fancyhf{}
\lfoot{Tobias Hilfiker}
\cfoot{Seite \thepage \hspace{1pt} von \pageref{LastPage}}
\rfoot{\today}

% Methadata
\title{
    \includegraphics{media/frontpage.jpg}
    \begin{center}
        Modul 152 \\
        Multimediainhalte in einen Webauftritt integrieren\\
        Storyboard
    \end{center}}
\author{Tobias Hilfiker}
\date{\today}

% Document
\begin{document}

    %Kapitel Frontpage
    \begin{titlingpage}
        \maketitle
    \end{titlingpage}
    \pagebreak

    %Kapitel Table of contents
    \tableofcontents
    \pagebreak

    %Kapitel 1 - Einleitung


    \section{Einleitung}
    Im Rahmen des Modul 152 darf ein E-Portfolio inklusive vorgehendem Storyboard erstellt werden.
    Dieses Storyboard soll eine Vorbereitung für das E-Portfolio sein, welches in einer separaten
    Dokumentation beschrieben wird.

    %Kapitel 1.1 - Ziel dieses Storyboards

    \subsection{Ziel dieses Storyboards}
    Mit diesem Storyboard habe ich das Ziel, die grundlegenden Konzepte für das E-Portfolio zu definieren.
    Dazu gehören Farbschemen, Logo, aber auch ein Mockup der Webseite.
    Folgende Fragen sollen bis zum Ende des Storyboards geklärt sein:
    \begin{itemize}
        \item Wie soll das Logo des E-Portfolios aussehen?
        \item Welche Farben sollen wo auf der Webseite verwendet werden? \textrightarrow Farbschema erstellen
        \item Wie soll die grundsätzliche Struktur des E-Portfolios aussehen?
        \textrightarrow Erstellung eines Mockups
        \item Welche Technologien möchte ich für die Erstellung der Webseite verwenden?
    \end{itemize}
    \\
    Neben den oben genannten Punkten möchte ich zudem bereits ein Bild, welches zum Thema passt bearbeiten.
    Um dieses Bild zu machen, informiere ich mich vorher in den Unterrichtsunterlagen und im Internet, welche
    Punkte beachtet werden müssen, um ein gutes Bild zu schiessen.
    Auch bei der Bildbearbeitung werde ich ähnlich vorgehen. Nachdem die Bildbearbeitung gelungen ist, werde
    ich eine Schritt-für-Schritt Anleitung zu dieser Bildbearbeitungsmethode ins Storyboard schreiben.

    %Kapitel 1.2 - Spezielles

    \subsection{Spezielles}
    Um das Storyboard und die Dokumentation zum E-Portfolio zu schreiben, verwende ich \latex. Das ist
    eine Skriptbasierte Dokumentationssprache, ähnlich wie Markdown. Primär gibt es zwei Unterschiede zu
    Markdown:
    \begin{enumerate}
        \item \latex Dateien müssen Kompiliert werden. Dazu gibt es verschiedene Compiler, welche aus den
        .tex-Dateien dann z.B. ein Pdf kompilieren.
        \item In ein \latex Dokument können beliebig viele Packages eingebunden werden. Das bedeutet, dass
        Elemente wie Diagramme, oder eigene Formatierungen einfach in das gleiche Dokument eingebunden werden.
    \end{enumerate}

    \pagebreak
    %Kapitel 2 - Thema


    \section{Thema}

    %todo kleiner Text

    %Kapitel 2.1 - Wahl des Themas

    \subsection{Wahl des Themas}

    Ich habe das Thema Curling ausgewählt, da ich im Winter selbst Curling spiele. Dies tue ich bereits in der
    9. Saison in einem eingeschworenen Juniorenteam in St. Gallen.
    Meiner Meinung nach ist Curling eine sehr spannende Sportart, man muss beim Spielen sehr viele verschiedene
    Situationen abwägen, sich für die beste entscheiden und danach möglichst viele Faktoren zu seinen Gunsten
    beeinflussen. \\
    Die meisten Personen kennen Curling als Sportart und wirken auch meist sehr interessiert. Die meisten haben
    aber meist keine Ahnung von den Begriffen und den Regeln, was das zuhören auf den Zuschauerbänken für mich
    meist sehr lustig gestaltet \smiley. Daher habe ich mit diesem E-Portfolio das Ziel, den Besuchern
    die Regeln und die Begrifflichkeiten etwas näher zu bringen.

    %Kapitel 3 - Visualisierung E-Portfolio


    \section{Visualisierung E-Portfolio}

    %Kapitel 3.1 - Menüstruktur

    \subsection{Menüstruktur}
    Als Menü-Navigation habe ich mich für eine Navigation am oberen Teil der Webseite entschieden.

    %Kapitel 3.2 - Farbwahl

    \subsection{Farbwahl}
    Um eine Farbauswahl habe ich mal simpel nach Farbpaletten im Internet gesucht. Dabei ist mir die Seite
    \url{colorhunt.co} aufgefallen. Diese Webseite lässt User ihre eigenen Farbpaletten hochladen und mit
    anderen Usern teilen. Zudem habe ich Adobe Color ausprobiert, allerdings ist dies viel zu viel
    Funktionalität für dieses Projekt. Daher werde ich für das Storyboard die Plattform Colorhunt verwenden.\\
    https://colorhunt.co/palette/2c3333395b64a5c9cae7f6f2\\
    https://colorhunt.co/palette/e8f9fd79dae80aa1dd2155cd \textleftarrow favorit\\
    https://colorhunt.co/palette/002b5b2b4865256d858fe3cf\\
    https://colorhunt.co/palette/2f8f9d3bacb682dbd8b3e8e5\\

    \pagebreak %TODO remove when Chapter 3.2 is finished

    %Kapitel 3.3 - Mockups

    \subsection{Mockups}
    Um die Webseite grob zu designen wollte ich ein Mockup-Programm nutzen. Nachfolgend werde ich Dokumentieren,
    wie ich das Mockup-Programm ausgewählt habe, wie ich die Mockups gestaltet habe und zu guter letzt auch
    die Mockups selbst in die Dokumentation einfügen.

    %Kapitel 3.3.1 - Auswahl des Mockup-Programms
    \subsubsection{Auswahl des Mockup-Programms}
    Um herauszufinden, welches Mockup-Programm ich während des Storyboards und E-Portfolio verwende, habe ich
    mich im Internet Informiert, welche verschiedenen Mockup-Programme dass es gibt. Folgende Programme habe
    ich gefunden und genauer analysiert:

    \begin{itemize}
        \item Balsamiq Mockups 3
        \item Moqups.com
        \item Adobe xD
        \item Mockplus.com
    \end{itemize}\\
    Um mich auf ein Tool festzulegen, habe ich Kriterien definiert, und anhand dieser die Tools bewertet. So
    kann ich anschliessend in der Entscheidungsmatrix ablesen, welches Mockup-Programm ich nachher effektiv
    nutze. Folgende Kriterien habe ich vorgängig definiert:

    \begin{itemize}
        \item Erfahrung \textrightarrow Wie viel Erfahrung habe ich bereits mit diesem Tool?
        \item Preis \textrightarrow Wie viel kostet das Tool (je weniger, desto besser)
        \item Funktionalität \textrightarrow Bietet mir das Tool alle Funktionalitäten die ich benötige?
    \end{itemize}
    Um die Entscheidungsmatrix auszufüllen, werde ich eins bis fünf Punkte vergeben. Ein Punkt ist dabei das tiefste
    und fünf das höchste. Das Tool mit den total meisten Punkten gewinnt am Schluss.\\

    \begin{center}
        \begin{tabular}{ | p{4cm} | p{2.5cm} | p{2.5cm} | p{3cm} | p{2.5cm} | }
            \hline
            & \textbf{Erfahrung} & \textbf{Preis} & \textbf{Funktionalität} & \textbf{Total Score} \\ \hline
            \textbf{Balsamiq Mockups} & 3                  & 2              & 4                       & 10                   \\ \hline
            \textbf{Moqups.com}       & 1                  & 2              & 3                       & 6                    \\ \hline
            \textbf{Adobe xD}         & 5                  & 4              & 5                       & 14                    \\ \hline
            \textbf{Mockplus.com}     & 1                  & 3              & 5                       & 9                    \\ \hline
        \end{tabular}
    \end{center}
    \\
    \textbf{Begründung der Entscheidung}\\
    Anhand der obigen Entscheidungsmatrix habe ich mich für das Programm Adobe xD entschieden. Dies hat mehrere
    Gründe. Einerseits habe ich in vergangenen Projekten bereits mit xD gearbeitet, andererseits bietet es
    mir neben den Mockups auch die Funktionalität, einen Prototyp zu gestalten. Zudem habe ich durch eine
    Adobelizenz der Schule xD bereits lizenziert, was bei allen anderen Programmen nicht der Fall ist.

    %Kapitel 3.3.2 - Design der Mockups
    \subsubsection{Design der Mockups}

    %Kapitel 3.3.3 - Mockups
    \subsubsection{Mockups}
    - Bilder der Mockups einfügen

    %Kapitel 4 - Bildbearbeitung


    \section{Bildbearbeitung}

    %Kapitel 4.1 - Bildbearbeitungsmethode

    \subsection{Bildbearbeitungsmethode}

    %Kapitel 5 - Logo


    \section{Logo}

    %Kapitel 5.1 - Idee

    \subsection{Idee}

    %Kapitel 6 - Technisches Konzept E-Portfolio


    \section{Technisches Konzept E-Portfolio}
    -Welche Technologien werden verwendet

    %Kapitel 7 - Reflexion


    \section{Reflexion}

    %Kapitel 7.1 - Bildbearbeitung

    \subsection{Bildbearbeitung}

    %Kapitel 7.2 - Logodesign

    \subsection{Logodesign}

    %Kapitel 8 - Verzeichnisse


    \section{Verzeichnisse}

    %Kapitel 8.1 - Abbildungsverzeichnis

    \subsection{Abbildungsverzeichnis}

    %Kapitel 8.2 - Glossar

    \subsection{Glossar}

\end{document}