%! Author = hilfiker
%! Date = 21.07.2022

% Preamble
\documentclass[11pt]{article}

% Packages
\usepackage{amsmath} % Formulas in document
\usepackage{pdflscape} % Landscaped pages in Pdf
\usepackage[a4paper, margin=0.8in]{geometry} % Set the margin and size of a page
\usepackage[hidelinks]{hyperref} % Remove Boxes around Hyperlinks
\usepackage{lastpage} % Custom page numbering
\usepackage{fancyhdr}
\usepackage{graphicx}
\usepackage{tcolorbox} % Custom page numbering

\usepackage{titling}
\usepackage{textcomp}
\renewcommand\maketitlehooka{\null\mbox{}\vfill}
\renewcommand\maketitlehookd{\vfill\null}

% Configure page numbering & footer
\pagestyle{fancy}
\fancyhf{}
\lfoot{Tobias Hilfiker}
\cfoot{Seite \thepage \hspace{1pt} von \pageref{LastPage}}
\rfoot{\today}

% Methadata
\title{
    \includegraphics{media/frontpage.jpg}
    \begin{center}
        Modul 152 \\
        Multimediainhalte in einen Webauftritt integrieren\\
        Storyboard
    \end{center}}
\author{Tobias Hilfiker}
\date{\today}

% Document
\begin{document}

    %Kapitel Frontpage
    \begin{titlingpage}
        \maketitle
    \end{titlingpage}
    \pagebreak

    %Kapitel Table of contents
    \tableofcontents
    \pagebreak

    %Kapitel 1 - Einleitung
    \section{Einleitung}
    Im Rahmen des Modul 152 darf ein E-Portfolio inklusive vorgehendem Storyboard erstellt werden.
    Dieses Storyboard soll eine Vorbereitung für das E-Portfolio sein, welches in einer separaten
    Dokumentation beschrieben wird.

    %Kapitel 1.1 - Ziel dieses Storyboards
    \subsection{Ziel dieses Storyboards}
    Mit diesem Storyboard habe ich das Ziel, die grundlegenden Konzepte für das E-Portfolio zu definieren.
    Dazu gehören Farbschemen, Logo, aber auch ein Mockup der Webseite.
    Folgende Fragen sollen bis zum Ende des Storyboards geklärt sein:
    \begin{itemize}
        \item Wie soll das Logo des E-Portfolios aussehen?
        \item Welche Farben sollen wo auf der Webseite verwendet werden? \textrightarrow Farbschema erstellen
        \item Wie soll die grundsätzliche Struktur des E-Portfolios aussehen?
                \textrightarrow Erstellung eines Mockups
        \item Welche Technologien möchte ich für die Erstellung der Webseite verwenden?
    \end{itemize}
    \\
    Neben den oben genannten Punkten möchte ich zudem bereits ein Bild, welches zum Thema passt bearbeiten.
    Um dieses Bild zu machen, informiere ich mich vorher in den Unterrichtsunterlagen und im Internet, welche
    Punkte beachtet werden müssen, um ein gutes Bild zu schiessen.
    Auch bei der Bildbearbeitung werde ich ähnlich vorgehen. Nachdem die Bildbearbeitung gelungen ist, werde
    ich eine Schritt-für-Schritt Anleitung zu dieser Bildbearbeitungsmethode ins Storyboard schreiben.

    %Kapitel 1.2 - Spezielles
    \subsection{Spezielles}
    Um das Storyboard und die Dokumentation zum E-Portfolio zu schreiben, verwende ich \latex. Das ist
    eine Skriptbasierte Dokumentationssprache, ähnlich wie Markdown. Primär gibt es zwei Unterschiede zu
    Markdown:
    \begin{enumerate}
        \item \latex Dateien müssen Kompiliert werden. Dazu gibt es verschiedene Compiler, welche aus den
        .tex-Dateien dann z.B. ein Pdf kompilieren.
        \item In ein \latex Dokument können beliebig viele Packages eingebunden werden. Das bedeutet, dass
        Elemente wie Diagramme, oder eigene Formatierungen einfach in das gleiche Dokument eingebunden werden.
    \end{enumerate}

    %Kapitel 2 - Thema
    \section{Thema}

    %todo kleiner Text

    %Kapitel 2.1 - Wahl des Themas
    \subsection{Wahl des Themas}

    Ich habe das Thema Curling ausgewählt, da es bereits seit vielen Jahren mein Hobby ist. Jeden Winter
    von ca. September bis April spiele ich in einem Juniorenteam der Liga C in St.Gallen.

    %Kapitel 3 - Visualisierung E-Portfolio
    \section{Visualisierung E-Portfolio}

    %Kapitel 3.1 - Menüstruktur
    \subsection{Menüstruktur}
    Als Menü-Navigation habe ich mich für eine Navigation am oberen Teil der Webseite entschieden.

    %Kapitel 3.2 - Farbwahl
    \subsection{Farbwahl}

    %Kapitel 3.3 - Mockups
    \subsection{Mockups}
    - Mockups mit Adobe xd -> Wieso
    - Bilder der Mockups einfügen

    %Kapitel 4 - Bildbearbeitung
    \section{Bildbearbeitung}

    %Kapitel 4.1 - Bildbearbeitungsmethode
    \subsection{Bildbearbeitungsmethode}

    %Kapitel 5 - Logo
    \section{Logo}

    %Kapitel 5.1 - Idee
    \subsection{Idee}

    %Kapitel 6 - Technisches Konzept E-Portfolio
    \section{Technisches Konzept E-Portfolio}
    -Welche Technologien werden verwendet

    %Kapitel 7 - Reflexion
    \section{Reflexion}

    %Kapitel 7.1 - Bildbearbeitung
    \subsection{Bildbearbeitung}

    %Kapitel 7.2 - Logodesign
    \subsection{Logodesign}

    %Kapitel 8 - Verzeichnisse
    \section{Verzeichnisse}

    %Kapitel 8.1 - Abbildungsverzeichnis
    \subsection{Abbildungsverzeichnis}

    %Kapitel 8.2 - Glossar
    \subsection{Glossar}

\end{document}